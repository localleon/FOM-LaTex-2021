% Opptionen für Biblatex
\ExecuteBibliographyOptions{%
  giveninits=false,
  sorting=nty, % sortierung nach name, titel , jahr 
  isbn=false,
  url=true,
  doi=true,
  eprint=false,
  maxbibnames=7, % Alle Autoren (kein et al.)
  maxcitenames=2, % et al. ab dem 3. Autor
  mincitenames=1,
  backref=false, % Rückverweise auf Zitatseiten
  bibencoding=utf8, % wenn .bib in utf8, sonst ascii
  bibwarn=true, % Warnung bei fehlerhafter bib-Datei
  maxbibnames=999,
  date=iso,
  seconds=true, %werden nicht verwendet, so werden aber Warnungen unterdrückt.
  urldate=iso,
  dashed=false,
  autocite=inline,
  useprefix=true, % 'von' im Namen beachten (beim Anzeigen)
  mincrossrefs = 1,
  defernumbers=true % Geteilte Literaturverzeichnisse gemeinsam nummerieren
}%

% et al. an Stelle von u.a.
\DefineBibliographyStrings{ngerman}{
  andothers = {{et\,al\adddot}},
}

%% et al. anstatt u. a. bei mehr als drei Autoren.
\DefineBibliographyStrings{ngerman}{
  andothers = {{et\,al\adddot}},
}
\DefineBibliographyStrings{english}{
  andothers = {{et\,al\adddot}},
}


% Klammern um das Jahr in der Fußnote
% Package biblatex Warning: Macro 'cite:labelyear+extrayear' undefined. Using \newbibmacro on input line 31.
%\renewbibmacro*{cite:labelyear+extrayear}{%
%  \iffieldundef{labelyear}
%    {}
%    {\printtext[bibhyperref]{%
%       \mkbibparens{%
%         \printfield{labelyear}%
%         \printfield{extrayear}}}}}

% Package biblatex Warning: Macro 'cite:title' undefined. Using \newbibmacro on input line 39.
%\renewbibmacro*{cite:title}{%
%  \printtext[bibhyperref]{%
%    \printfield[citetitle]{labeltitle}%
%    \setunit{\addcomma\space}%
%    \printdate}}

\DeclareNameFormat{last-first}{%
  \iffirstinits
  {\usebibmacro{name:family-given}
    {\namepartfamily}
    {\namepartgiveni}
    {\namepartprefix}
    {\namepartsuffix}
  }
  {\usebibmacro{name:family-given}
    {\namepartfamily}
    {\namepartgiven}
    {\namepartprefix}
    {\namepartsuffix}
  }%
  \usebibmacro{name:andothers}}

% Alternative Notation der Fußnoten
% Zeigt sowohl den Nachnamen als auch den Vornamen an
% Beispiel: \fullfootcite[Vgl. ][Seite 5]{Tanenbaum.2003}
\DeclareCiteCommand{\fullfootcite}[\mkbibfootnote]
{\usebibmacro{prenote}}
{\usebibmacro{citeindex}%
  \printnames[sortname][1-1]{author}%
  \addspace (\printfield{year})}
{\addsemicolon\space}
{\usebibmacro{postnote}}

%Autoren (Nachname, Vorname)
\DeclareNameAlias{default}{family-given}